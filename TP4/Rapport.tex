\documentclass{report}
\usepackage{MCC}

\def\footauthor{Victor COLEAU \& Thomas COUCHOUD}
\title{Développement mobile - TP4 \& TP5}
\author{Victor COLEAU\\\texttt{victor.coleau@etu.univ-tours.fr}\\Thomas COUCHOUD\\\texttt{thomas.couchoud@etu.univ-tours.fr}}

\begin{document}
	\mccTitle
	\tableofcontents
	\chapter{Questions}
		\section{Installation de l'APK}
			Nous pouvons grâce à \cbo{adb devices} obtenir la liste des périphériques connectés.
			Puis l'APK est installée avec \cbo{adb install <path>}.
			
		\section{Premier lancement}
			L'application permet d'obtenir la vidéo de la caméra a 30 fps.
			
			\subsection{Cycle de vie}
				L'application passe par les états suivants:
				\begin{easylist}
					@ Création d'une instance de l'Activity
					@ Appel de onCreate permettant d'initialiser nos différentes vues
					@ Appel de onResume démarrant l'accès à la caméra
				\end{easylist}
				
				Lors du passage de l'application en tâche de fond, onPause est appelé ce qui a pour effet d'arrêter les mises à jours depuis la caméra.
				De manière similaire lors du passe au premier plan de l'application, onResume est rappelée et relance la caméra.
				
				Lors de la destruction de l'application, onDestroy est appelé et détruit nos différents objets.
				
				Lors de l'appel à onResume, on initialise openCV. Deux cas peuvent se présenter:
				\begin{easylist}[itemize]
					@ OpenCV est fourni dans le package de l'application (notre cas), et on a juste à appeler le callback pour démarrer la vue
					@ OpenCV n'est pas dans le package de l'application, dans ce cas on demande à OpenCV de l'initialiser à partir d'OpenCV Manager (qui se chargera d'appeler le callback pour démarrer la vue).
				\end{easylist}
			
			\subsection{Gestion camera}
				onCameraViewStarted est appelée lorsque la connexion à la caméra a été effectuée.
				Cela nous permet notamment de récupérer la taille de celle-ci pour initialiser nos différentes variables.
				
				onCameraFrame est appelée lorsque une nouvelle image de la caméra est prête à être affichée.
				A ce moment nous pouvons donc la modifier si nécessaire.
		\section{Gradient JAVA}
			Avec l'opération du gradient, nous devrions observer les contours des différents éléments de l'image.
			
			Avec cette opération supplémentaire, nous arrivons aux alentours de 2-3fps.
			
		\section{Sobel JAVA}
			Le filtre de sobel étant encore plus gourmand en ressources que le filtre gradient, nous arrivons avec un résultat d'environ 0.02fps avec un masque de convolution $3\times 3$:
			$$\begin{bmatrix}
				0 & -1 & 0\\
				-1 & 0 & 1\\
				0 & 1 & 0
			\end{bmatrix}$$
		
		\section{JNI}
			Le nom de la fonction dans le code CPP est: Java\_com\_example\_polytech\_app\_MainActivity\_gauss.
			
			La classe MainActivity possède cette méthode.
			
			Le mot clef native permet de dire à Java que l'implémentation de la fonction se trouve dans une librairie externe en C.
			
			Le System.loadLibrary() permet de charger cette librairie afin d'obtenir l'implémentation de notre fonction.
			
		\section{Résumé des FPS}
			\begin{tabular}{|c|c|c|c|c|}
				\hline
				Méthode & JAVA (Samsung) & Java (HP) & C++ (Samsung) & C++ (HP)\\\hline
				Aucun & 30 & 28 & - & -\\\hline
				Gradient & 2.83 & 0.02 & 3.99 & 2.8\\\hline
				Sobel & 0.9 & 0.01 & 1.7 & 1.9\\\hline
			\end{tabular}
			
			De manière générale avec le même code transposé en C++, nus arrivons à observer que le traitement se fait plus rapidement.
			
			Dans le cas des tablettes Samsung la différence reste peu percevable.
			En revanche les tablettes HP bénéficient d'un gros boost grâce à ce passage.
\end{document}
