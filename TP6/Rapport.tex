\documentclass{report}
\usepackage{MCC}

\def\footauthor{Victor COLEAU \& Thomas COUCHOUD}
\title{Développement mobile - TP6}
\author{Victor COLEAU\\\texttt{victor.coleau@etu.univ-tours.fr}\\Thomas COUCHOUD\\\texttt{thomas.couchoud@etu.univ-tours.fr}}

\begin{document}
	\mccTitle
	\tableofcontents
	\chapter{Questions}
		\section{Question 2}
			En executant la requête dans le thread UI, il ne se passe rien du tout même en ayant préciser que notre application a besoin des permissions internet.
			
		\section{Question 3}
			Avec le bypass cela fonctionne cependant cette pratique n'est pas conseillée.
			
		\section{Question 4}
			Le AsynTask permet aussi de réaliser ce que l'on veut tout en ayant un code propre.
\end{document}
