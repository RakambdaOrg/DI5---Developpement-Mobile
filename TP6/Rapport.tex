\documentclass{report}
\usepackage{MCC}

\def\footauthor{Victor COLEAU \& Thomas COUCHOUD}
\title{Développement mobile - TP6}
\author{Victor COLEAU\\\texttt{victor.coleau@etu.univ-tours.fr}\\Thomas COUCHOUD\\\texttt{thomas.couchoud@etu.univ-tours.fr}}

\begin{document}
	\mccTitle
	\tableofcontents
	\chapter{Questions}
		\section{Question 2}
			En executant la requête dans le thread UI, il ne se passe rien du tout même en ayant préciser que notre application a besoin des permissions internet.
			
		\section{Question 3}
			Avec le bypass cela fonctionne cependant cette pratique n'est pas conseillée.
			
		\section{Question 4}
			Le AsynTask permet aussi de réaliser ce que l'on veut tout en ayant un code propre.
			
		\section{Question 5}
			\img{start.png}{Ecran d'acceuil}{0.2}
			
			Au démarrage de l'application nous pouvons voir les éléments suivant:
			\begin{easylist}[itemize]
				@ Un champ de texte afin de renseigner l'adresse IP à chercher (si vide l'adresse IP de la requête sera utilisée).
				@ Un bouton << Random >> permettant d'afficher une adresse IP aléatoire.
				@ Un bouton << Start >> permettant de lancer la requête.
				@ Ainsi qu'un champ de texte, pour l'instant vide, où les résultats s'afficheront.	
			\end{easylist}

			Il est à noter que le champ de saisie est configuré en tant que nombre afin d'afficher un clavier adapté (chiffres et point)(inputType=phone).
			De plus un filtre est présent afin de valider ou non le format de l'adresse entrée.
			Si ce dernier est invalide la saisie n'est pas prise en compte.
			Cela est réalisé grace à un InputFilter.
			
			Lors du clic sur le bouton start, un tache asynchrone est démarrée afin d'effectuer la requête.
			
			\begin{figure}[H]
				\begin{minipage}{0.45\textwidth}
					\img{search.png}{Requête valide}{0.2}
				\end{minipage}
				\begin{minipage}{0.45\textwidth}
					\img{nodata.png}{Requête valide}{0.2}
				\end{minipage}
			\end{figure}
			
			Si les données reçues correspondent à une IP valide, on affiche ces dernières sur un fond vert.
			
			Si les données reçues correspondent à une IP indisponible, on affiche un message d'erreur sur fond rouge.
	

			\img{invalid.png}{Requête valide}{0.16}
			Si l'adresse de requêtage est mal formatée (partie manquante), on affiche un message d'erreur sur fond deep pink ainsi qu'un Toast.
			
			\subsection{Requête}
				Afin d'effectuer les requêtes vers l'api, nous avons créé une classe héritant de AsyncTask.
				Pour pouvoir écrire nos résultats dans notre activity de base, le constructeur prend en paramètre une TextView.
				
				Lors d'un appel à cette tache, la méthode doInBackground est appelée avec en paramètre l'URL à requêtes.
				Une fois la page récupérée, doInBackground retourne un String la représentant ce qui va permettre de la traiter dans onPostExecute.
				Dans cette dernière on parse le JSON obtenu et construisons un objet GeoIp qui est ensuite écrit dans le TextView passé dans le constructeur.
\end{document}
